\documentclass[12pt]{article}
\usepackage{amsfonts}
\usepackage{amsmath}
\usepackage{amsthm}
\usepackage[english]{babel}
\usepackage{booktabs}
\usepackage[labelfont=bf]{caption}
\usepackage{epigraph}
\usepackage{fullpage}
\usepackage{graphicx}
\usepackage[utf8]{inputenc}
\usepackage[numbers]{natbib}
\usepackage{parskip}
\usepackage{subcaption}

\usepackage[
  colorlinks=true,
  linkcolor=black,
  anchorcolor=black,
  citecolor=black,
  filecolor=black,
  menucolor=black,
  runcolor=black,
  urlcolor=black]{hyperref}

\newtheorem*{definition}{Definition}

\theoremstyle{definition}
\newtheorem{example}{Example}[section]

\DeclareMathOperator*{\argmin}{arg\,min}
\newcommand{\calN}{\mathcal{N}}
\newcommand{\hic}{\textrm{HIC}}

\title{Hierarchical Information Content}
\author{Awni Hannun\footnote{
  Send correspondence to
  \href{mailto:awni.hannun@gmail.com}{awni.hannun@gmail.com}}}

\date{\today}

\begin{document}
\maketitle

\begin{abstract}
  As simulations of digital life increase in scale, identifying interesting
  emergent structures and processes becomes a challenge.  Solving this will
  require quantitative criteria which align with the goals of the system
  designer. Such criteria are usually reduced to a measure of complexity. We
  argue that complexity growth is induced at the composition of order and
  disorder. Hierarchies built from such compositions lead to further increasing
  complexity. Based on these observations we define \emph{hierarchical
  information content} and motivate it as a measure of complexity. We
  demonstrate through quantitative examples and experiments on elementary
  cellular automata that hierarchical information content successfully
  distinguishes systems based on their complexity.
\end{abstract}

\section{Introduction}
\label{sec:intro}

A common goal in digital simulations of life is to induce the emergence of
certain structures or processes. These might include self-replicating or
autopoietic organisms, evolutionary processes, and intelligent behavior. Often
identifying the growth of these structures and processes is reduced to
identifying a growth in complexity. Hence, designers of artificial life
simulations are interested in the question of measuring complexity.

As artificial life simulations scale, quantitatively assessing the simulation's
success in meeting the designer's criteria is essential.  Identifying and
selecting amongst the design choices of a given simulation in a qualitative
manner is not scalable and likely to admit experimental bias. As an analogy,
imagine constructing a machine-learning model to learn a given task without an
objective function. Curiosity driven exploration and reinforcement learning can
result in the model learning to successfully complete the task.  However, from
the empiricists perspective, assessing the ability of the model at completing
the task is critical. At the very least, some criterion is needed for
evaluating the model, if not training it. Similarly, in artificial life, while
the measure of complexity is typically not used as an objective to guide the
active simulation, it is at least essential to evaluate the post hoc success of
the simulation.

Complexity is notoriously ill-defined, subjective, and difficult to measure
quantitatively~\citep{gell2002complexity, mitchell2009complexity,
wiesner2019measuring}. One of the challenges in measuring complexity is
identifying the ``Goldilocks zone'' between total order and total disorder
where complexity may reside (\emph{c.f.}
figure~\ref{fig:complexity_and_entropy}). Information theoretic measures such
as Shannon entropy~\citep{shannon1948} and Kolmogorov complexity (sometimes
called Algorithmic Information Content)~\citep{kolmogorov1965, solomonoff1964},
are actually measures of randomness. As figure~\ref{fig:complexity_and_entropy}
illustrates, these measures increase monotonically with the disorder of a
system.

\begin{figure}
\centering
\includegraphics[width=0.7\textwidth]{figures/complexity_and_entropy}
\caption{An illustration of the one-hump of complexity (solid line) between
order and disorder. Measures such as Shannon entropy or algorithmic
information content (AIC) (both illustrated by the dashed line) grow
monotonically with disorder.}
\label{fig:complexity_and_entropy}
\end{figure}

Because of this difficulty, many measures of complexity have been
proposed~\citep{lloyd2001measures}, all with differing trade-offs. Often they
fail the ``one-hump'' criterion~\citep{adami2002complexity} increasing with
order or disorder. This is typical of the aforementioned information theoretic
criteria (Shannon entropy and Kolmogorov complexity) and constructions which
use them~\citep{lloyd1988complexity}. Other measures which satisfy the one-hump
test are not easily made operational requiring either a large degree of
subjective input from the designer or are exceedingly difficult to
compute~\citep{crutchfield1989inferring, gell1996information,
grassberger1986toward}.

We propose yet another a measure complexity, Hierarchical Information Content
(HIC). The central tenet of HIC is that complexity growth is found at the
compositions of ordered and disordered systems. We elaborate and defend the
motivating principles of HIC in section~\ref{sec:finding_complexity}. We do not
intend for HIC to serve as an ersatz for complexity in a general sense, and in
fact we make make no further attempt to rigorously define complexity. Rather,
we propose HIC as a measure which quantitatively captures some of the essential
aspects of complex systems. We promptly disclose two limitations of HIC. First,
HIC is not intended to work well for every case, but to work well for many
cases that are of interest, particularly in simulations of artificial life.
Second, while HIC can be used with few assumptions, in many instances it may
gain utility from the subjective input by the designer of the system.

\section{Finding Complexity}

We hypothesize that complexity grows from the compositions of order and
disorder.  When a system composes an ordered structure from disordered
primitives, opportunity for complexity growth emerges. At a higher level many
ordered structures can become disordered. This leads to a hierarchy of levels
with varying degrees of order and disorder creating a complex macro structure.
We motivate the idea of inducing complexity from composing order and disorder
in the sequel, followed by a motivation of the importance of hierarchies. 

\subsection{Order from Disorder}

\citet{adami2002complexity} define two types of complexity: i) {\emph
structural complexity}, which can be thought of as complexity in space and ii)
{\emph process complexity}, which can be thought of as complexity in time.
Following \citet{schrodinger1944}, we consider two types of composition, the
first is order from order and the second is order from disorder.  \citet[chap.
6]{schrodinger1944} argues that living beings exhibit order from order. In an
abstract sense, an living organism ``feeds on negative entropy'' (\emph{i.e.}
order) to produce the necessary building blocks needed to reduce the organisms
natural tendency towards entropy. More concretely, organisms create highly
structured building blocks at the molecular level (DNA, RNA, proteins,
\emph{etc.}) by metabolising highly structured inputs. This process describes
the development of complexity through time. Hence complexity in time (process
complexity) is involves the composition of order from order. 

On the otherhand, complexity in space constructs higher-level structures from
lower level structures. If the structures are ordered throughout the levels of
the hierarchy, the global structure is unlikely to be complex. Some degree of
disorder should be present at various levels of the hierarchy. Thus complexity
in space (structural complexity) emerges in part from the composition of order
from disorder.

Given the distinctions between how complexity manifests in time and space two
separate measures may be appropriate. Here, we are concerned with the later --
measuring complexity in space. The nexus of complexity growth is found in the
composition of order from disorder. In order to continue to grow complexity a
structure must continue to increase the number of such compositions. A
hierarchy must develop where some levels are more ordered and some are more
disorderd. This it the primary motivation of our measure of complexity which we
define more rigorously in section~\ref{sec:hic}.

\subsection{Complexity in Hierarchies}

We now consider the importance of hierarchy in complexity. inducing composition
of order from disorder

\section{Hierarchical Information Content}
\label{sec:hic}

We construct a measure for the structural complexity of a system. As motivated
in section~\ref{sec:finding_complexity}, we want the the complexity to grow at
the compositions of ordered and disordered subsystems in a hierarchical
structure. We decompose the overall complexity as the sum of the complexity at
each composition.

Assume we are given a hierarchical system with $L$ levels. We construct the
global system complexity from the local complexity computed at each of the
$L-1$ compositions between consecutive levels. The local complexity measures
are given by the difference in mutual information of the system at level
$\ell$ and that of the subsytem at level $\ell - 1$. We denote by $X^{\ell}$ the
state variable at the $\ell$-th level of the hierarchy. Following
\citet{simon1991architecture}, we define the span of a level as the
number of elements from the $\ell - 1$ subsystem used to construct the state
$X^\ell$. For now we assume the span to be constant with a value of two for
every level. We later generalize to larger spans.

As an example, the state variable $X^\ell$ could be a molecule in an artificial
chemistry constructed from two atoms. Alternatively, $X^{\ell}$ could be an $n
\times n$ state of a cellular automata constructed from two $(\nicefrac{n}{2})
\times n$ sub-states. As an example from nature, $X^{\ell}$ could be a string
of $2^\ell$ base pairs of DNA constructed from two substrings of length $2^{\ell
- 1}$.

The complexity at the $\ell$-th composition is computed from the squared
difference of the mutual information of the span between level $\ell$ and level
$\ell + 1$:
\begin{equation}
    C_\ell = \left[ I(X^{\ell + 1}; Y^{\ell + 1}) - I(X^{\ell}; Y^{\ell}) \right]^2
\end{equation}
The mutual information can be computed from the Shannon entropy and conditional
entropy when $X^\ell$ is discrete or the differential entropy when $X^\ell$ is
continuous~\citep{cover1999elements}:
\begin{equation}
\label{eq:mutual_information}
I(X^\ell; Y^\ell) = H(X^\ell) - H(X^\ell \mid Y^\ell).
\end{equation}
The mutual information is large when the $\ell$-th level is highly ordered and
small when it is highly disordered. Hence, the values $C_\ell$ will be large
exactly at the transtion points between order and disorder (or vice-versa).  We
use the level complexities $C_\ell$ to construct the overall system complexity.

\begin{definition}[Hierarchical Information Content]
\label{def:hic}
  The Hierarchical Information Content (HIC) of a hierarchical system $S$ with
  $L$ levels is given by:
  \begin{equation}
    \label{eq:hic}
    \hic(S) = \sum_{l=1}^{L-1} C_\ell = \sum_{l=1}^{L-1} \left[ I(X^{\ell+1}; Y^{\ell+1}) - I(X^\ell; Y^\ell) \right]^2.
  \end{equation}
\end{definition}

\paragraph{Generalizing to larger spans:} Using a larger span requires
generalizing the mutual information to multiple variables. The interaction
information is one such generalization~\citep{mcgill1954multivariate}. The
interaction information of a set of variables is defined recursively as:
\begin{equation}
    \label{eq:interaction_information}
    I(X_1^\ell; \ldots; X_{n+1}^\ell) = I(X_1^\ell; \ldots; X_n^\ell) - I(X_1^\ell; \ldots; X_n^\ell \mid X_{n+1}^\ell),
\end{equation}
where the conditional mutual information is given by:
\begin{equation}
    \label{eq:conditional_mutual_information}
    I(X; Y \mid Z) = \sum_{Z} P(Z) I(X; Y).
\end{equation}

\paragraph{Estimating information criteria:} Computing the mutual information
in equation~\ref{eq:mutual_information} requires computing the entropy of the
state variables $H(X^\ell)$ and the conditional entropy $H(X^\ell \mid
Y^\ell)$. Estimating these quantities directly requires models for the
distributions $P(X^\ell)$ and $P(X^\ell \mid Y^\ell)$. For state variables with
a small domain these distributions can likely be estimated efficiently from
counts. However, for larger state variables sample efficiency becomes an issue.
In these cases more sophisticated density estimation techniques should be used.
Suitable techniques can be found in most modern statistical learning or machine
learning texts (see for example \citet{friedman2001elements}).

\paragraph{Specifying the hierarchy:} In some cases, the hierarchical structure
of the system may be self-evident. This might be the case if the system has a
natural tree-like structure. However, in the general case, both the boundaries
between levels and the submodules which make up the state at the next level up
must be specified. This is introduces a degree subjectivity to the HIC which
effects its ability to faithfully capture the complexity of the system.

As an example consider measuring the HIC of a sequence of DNA. We might define
the state $X^\ell$ to be a subsequence of $2^\ell$ nucleotides. On the other
hand there may be some natural structure in the DNA to take advantage of. For
example, codons are sequences of three nucleotides which each code for a
specific amino acid.  These would make a good definition for a state variable.

\subsection{Examples}

We attempt to build intuition for the definition of HIC through some simple
examples.

\begin{example}
  \label{ex:constant}
  As a first example, consider the state $S = [0, 0, \ldots, 0]$,
  a constant sequence of all $0$s. At any level for any neighborhood, the mutual
  information $I(X^\ell; Y^\ell) = 0$. Hence the terms $C_\ell = 0$ for all
  $\ell$ and the overall $\hic(S) = 0$.
\end{example}

\begin{example}
  \label{ex:uniform}
  Let $S = [x_1, x_2, \ldots]$ consist of a sequence of independent draws from a
  multinomial uniform distribution over $K$ categories. Assuming a span of $2$,
  the variable $X^\ell$ consists of $2^\ell$ of the $x_i$ primitives and hence
    can take on any of $K^{(2^\ell)}$ values. Consider the mutual information of level
  $\ell$:
  \begin{equation}
    I(X^\ell; Y^\ell) = H(X^\ell) - H(X^\ell \mid Y^{\ell}) = 2^\ell \log K - 2^\ell \log K = 0
  \end{equation}
  Hence all of the $C_\ell = 0$ and $\hic(S) = 0$.
\end{example}

As examples~\ref{ex:constant} and \ref{ex:uniform} show, the HIC behaves as
expected when the system exhibits complete order or complete disorder. In both
of these examples the HIC was zero because the mutual information at each level
was zero. In the following example, we see slightly more interesting behavior,
where the individual mutual informations are not zero, but the resulting HIC
is.

\begin{example}
  \label{ex:repeats}
  Let $S = [0, 1, 0, 1, \ldots]$ consist of a sequence of alternating zeros and
  ones.

  The first level $X^1$ takes on the two values $\{0, 1\}$ with equal
  probability hence $H(X^1) = \log 2$. However given the neighbor $Y^1$ the
  value of $X^1$ is deterministic and hence $H(X^1 \mid Y^1) = 0$. Thus the
  overall mutual information is $I(X^1; Y^1) = \log 2$.

  The second level $X^2$ takes on two values $\{[0, 1], [1, 0]\}$ with
  equal probability, and we have $H(X^2) = \log 2$. Again $X^2$ is fully
  determined by $Y^2$ thus $H(X^2 \mid Y^2) = 0$ and $I(X^2, Y^2) = \log 2$.
  Combining these to produce $C^2$ yields:
  \begin{equation}
    C_2 = (I(X^2; Y^2) - I(X^1; Y^1))^2 = (\log 2 - \log 2)^2 = 0.
  \end{equation}
  The above argument generalizes to all levels of the hierarchy so $\hic(S) = 0$.
\end{example}

In example \ref{ex:repeats} we used overlaping windows to construct $X^2$. If
we had chosen disjoint windows the HIC would be small but nonzero. Depending on
the problem this may be desirable in that the alternating seuqence should be
considered slightly more complex than a constant sequence. This is where
subjectivitity comes in. The designer should select from these choices
based on the setting at hand.

\subsection{Comparison to Statistical Complexity}
\label{statistical_complexity}

The statistical complexity of \citet{crutchfield1989inferring} measures the
complexity of a system by observing the number of distinct future distributions
given the present and past states.  Statistical complexity infers a set of
``causal states'' by collapsing the state variables representing the current
and past time into equivalence classes which all lead to indistinguishable
future distributions. The entropy of the distribution of these causal states
then serves as a measure of the complexity of the process. For futures which
are ordered or unchanging, the system will have one or a small number of causal
states and the entropy will be small. Similarly, for futures which are
uniformly random, the future distributions will be indistinguishable given the
past and present and the states will similarly collapse into a single or small
number of causal states. Thus random processes will also have low statistical
complexity.

Let $X_t$ represent the state variable at the $t$-th time step for a system
with $N$ state variables per time step. The past light cone
$\mathcal{X}_{\textrm{P}}$ of $X_t$ with a history of $T_{\textrm{P}}$ time
steps is given by:
\begin{equation}
    \mathcal{X}_{\textrm{P}} =
        \{X_{i, j} \mid i = t - 1, \ldots, t - T_{\textrm{P}}
            \;\;\land\;\; j = 1, \ldots, N
            \;\;\land\;\; X_{i, j} \rightarrow X_t \},
\end{equation}
where $A \rightarrow B$ means the state $A$ can influence $B$ (\emph{i.e.} $B$
is a function of $A$). The future light cone $\mathcal{X}_{\textrm{F}}$ of
$X_t$ with a future of $T_{\textrm{F}}$ time steps is similarly defined as:
\begin{equation}
    \mathcal{X}_{\textrm{F}} =
        \{X_{i, j} \mid i = t + 1, \ldots, t + T_{\textrm{F}}
            \;\;\land\;\; j = 1, \ldots, N
            \;\;\land\;\; X_t \rightarrow X_{i, j} \}.
\end{equation}
In words, the past light cone of $X_t$ is the set of states which can influence
$X_t$ and the future light cone is the set of states which can be influenced by
$X_t$.

Each past light cone and state variable $X_t$ defines a conditional
distribution over future light cones $P(\mathcal{X}_{\textrm{F}} \mid
\mathcal{X}_{\textrm{P}}, X_t)$. We can measure the similarity between to such
conditional distributions using a statistical divergence $D(P_i \| P_j)$. A
causal state $C$ represents a set of states $C = \{(\mathcal{X}_{\textrm{P}},
X_t)\}$ where $D(P_i \| P_j) = 0$ for every pair $(\mathcal{X}_{\textrm{P}},
X_t)_i$ and $(\mathcal{X}_{\textrm{P}}, X_t)_j$ in $C$. In other words, the
distribution of future light cones of $(\mathcal{X}_{\textrm{P}}, X_t)_i$ and
$(\mathcal{X}_{\textrm{P}}, X_t)_j$ are indistinguishable. In practice, we
relax $D(P_i \| P_j) < \tau$ for some predefined threshold due to finite
sample sizes. Furthermore, we only require that $D(P_i \| P_j) < \tau$ be
satisfied for a single $(\mathcal{X}_{\textrm{P}}, X_t)_j$ in $C$ instead of
all such states.

Unlike HIC, statistical complexity is a measure of process complexity. We can
use it to observe how the complexity of a system changes over time. We can also
use statistical complexity as a measure of the global complexity of the system
by applying it to every state at every time step; as we do in this work.

Perhaps more importantly, statistical complexity can be computed in many
cases without much difficulty. The distributions of future light cones given
past light cones and state variables can be estimated for reasonable values of
$T_\textrm{P}$ and $T_\textrm{F}$. \citet{shalizi2004quantifying} estimate the
local statistical complexity of circular cellular automata and show
qualitatively that the measure coorelates well to with the perceived complexity
of the automata.

\begin{figure}[ht]
\centering
\includegraphics[width=0.7\textwidth]{figures/eca_images_and_complexity}
\caption{Example ECA rules from the four Wolfram classes. Each ECA has a state
    size of $N=200$ states and is run for $400$ steps. The latter $200$ steps
    are visualized above. We compute the HIC and statistical complexity (SC)
    for each ECA.}
\label{fig:eca_images_and_complexity}
\end{figure}

\section{Experiments}

\subsection{Experimental Setting}

We examine HIC empirically using elementary cellular automata (ECA). We also
compare HIC to statistical complexity for various ECA. Cells in an ECA
can take on one of two states. The update rule for a given state depends on the
state at the previous time step along with its two neighbors; hence there are
$2^{(2^3)} = 256$ possible ECAs. \citet{wolfram1983} proposed a four class
system to categorize ECA rules by their typical behavior. Class 1 ECAs converge
to a constant state, class 2 ECAs tend to exhibit periodic oscillations, class
3 ECAs display random-like behavior, and class 4 ECAs show complex behavior.
Complex behavior of an ECA is not well defined and is determined mostly through
inspection. The complex ECAs usually have various high-level structures
propagating through time with some degree of apparent randomness interspersed
between. Despite the simplicity of the rules, ECAs can yield astonishingly
complex behavior. Rule 110, for example, has been shown to be computationally
universal~\citep{cook2004universality}. We follow the Wolfram classification
given by~\citet[table 2]{martinez2013note} to classify all 256 ECA rules.

For each ECA we use a state size of $N$. The information in the initial state
requires at least $N/2$ updates to propagate across the complete state space.
We update the ECA for $2N$ time steps and discard the first $N$ which provides
a buffer for information in the initial state to fully propagate. We use a
circular update so the cells at the edges are influenced by neighbors at the
opposite edge of the state.  Because of this, rules which propagate information
in a single direction can influence the full state space after $N$ updates.

We use the remaining $N \times N$ cells to compute the HIC. At each level we
set $X^\ell$ and $Y^\ell$, the variables used to compute the mutual information
in equation~\ref{eq:mutual_information}, to be neighboring sequences each
consisting of $\ell$ cells. So for the first level ($\ell = 1$) $X$ and $Y$ are
single and consecutive cells, for $\ell = 2$ the variables are consecutive but
disjoint pairs of cells, and so on. We estimate the distributions $P(X^\ell)$
and $P(X^\ell \mid Y^\ell)$ used to compute the mutual information at each
level from counts over the states.

To construct the causal states for statistical complexity, we use the
Kullback-Leibler divergence with a threshold of $1.0$. We use a past light cone
with $T_{\textrm{P}}=2$ and a future light cone with $T_{\textrm{F}} = 3$.  For
each time step the number of states in the light cone grows by two. For
example, the third time step of the future light cone contains five cells. As
with HIC we estimate the distributions using counts over the states.

\begin{figure}[ht!]
\centering
\begin{subfigure}{\textwidth}
    \centering
    \includegraphics[width=0.8\textwidth]{figures/hic_by_class}
    \caption{Hierarchical information content (HIC)}
    \label{fig:hic_by_class}
\end{subfigure}
\begin{subfigure}{\textwidth}
    \centering
    \includegraphics[width=0.8\textwidth]{figures/sc_by_class}
    \caption{Statistical complexity (SC)}
    \label{fig:sc_by_class}
\end{subfigure}
\caption{The median and 95\% confidence intervals for the HIC for each Wolfram
    class. We compute these by randomly sampling 100 rules with random initial
    states and computing the HIC of each. For each class, the first twenty
    trials within one standard deviation of the median are plotted.}
\label{fig:complexity_by_class}
\end{figure}


\subsection{Comparing HIC to Statistical Complexity}

Figure~\ref{fig:eca_images_and_complexity} shows a canonical ECA rule from each
class: rule 128 is class 1, rule 2 is class 2, rule 30 is class 3, and rule 110
is class 4. For these simulations we use $N\!=\!200$ and an initial state of all
inactive cells (zeros) with a single active cell (one) at the center. We
compute the HIC over $L\!=\!3$ levels. The HIC is given for each rule above the
corresponding image. Note that HIC correctly distinguishes rule 110 as being
the most complex ECA from the remainder, which have HICs very close to zero.
Note in particular that the HIC of the class 3 ECA (rule 30) is even smaller
than the corresponding class 2 ECA (rule 2). In this case HIC peaks in between
order and disorder, as desired. We also see that the statistical complexity
(SC) correctly orders the rules from each class. However, unlike HIC the class
3 ECA (rule 30) has a fairly large statistical complexity. This is the case
because the ECA exhibits randomness over the spatial dimension but has some
structure over the temporal dimensions.

Figure~\ref{fig:complexity_by_class} demonstrates that both HIC and statistical
complexity can distinguish between the four Wolfram complexity classes. These
results are obtained by sampling $100$ rules for each of the four ECA classes.
Each rule is run from a random initial state for $1000$ steps. We use a state
size of $N\!=\!500$ and hence a $500 \times 500$ grid to compute the
corresponding complexity measure. For these experiments we use $L=6$ levels for
the HIC. The class medians are well separated by the median HIC
(fig.~\ref{fig:hic_by_class}) with the 95\% confidence interval of class 4
distinct from that of any other class.

We also plot in figure~\ref{fig:hic_by_class} the first twenty trials for each
class with an HIC within one standard deviation of the median. Rules from ECA
class 2 have the widest range of HIC. Qualitative inspection of the high HIC
results shows many cases where the resulting state is simply shifted to the
left or right by the rule. The state exhibits a simple periodic pattern over
time but because of the random initialization, the pattern over space does not
appear to be simple. This is one failure mode of HIC -- it does not
capture the simplicity over the time dimension given that it is only applied to
the space dimension.

The statistical complexity medians are less well separated than those computed
from the HIC. On the other hand, the statistical complexity
(fig.~\ref{fig:sc_by_class}) has narrower 95\% confidence intervals for all
classes. Overall, both measures are quite good at distinguish complex class 4
ECA from the rest. Both measures could be useful depending on whether one
intends to measure process or structural complexity. The two may also be
combined to produce a single more comprehensive measure.

\subsection{Number of Levels}

\begin{figure}[t]
\centering
\begin{subfigure}{0.45\textwidth}
  \includegraphics[width=1.0\textwidth]{figures/hic_vs_num_levels_size_200}
  \caption{State size $N = 200$}
  \label{fig:hic_levels_200}
\end{subfigure}
\begin{subfigure}{0.45\textwidth}
  \includegraphics[width=1.0\textwidth]{figures/hic_vs_num_levels_size_500}
  \caption{State size $N = 500$}
  \label{fig:hic_levels_500}
\end{subfigure}
\caption{We show the HIC versus the number of levels $L$ used in the
    computation. We compute the HIC for one rule from each of the four Wolfram
    classes. Each ECA is run for $2N$ time steps and the HIC is computed from
    the latter half.}
\label{fig:hic_vs_levels}
\end{figure}

Using the same four rules (class 1 rule 128, class 2 rule 2, class 3 rule 30
and class 4 rule 110), we observe the effect of the number of levels on the HIC
in figure~\ref{fig:hic_vs_levels}. Figure~\ref{fig:hic_levels_200} uses a state
size of $N\!=\!200$, and figure~\ref{fig:hic_levels_500} uses a state size of
$N\!=\!500$. In both figures, we see that the class 1 and 2 rules (the curves
overlap at zero) do not grow with an increase in the number of levels. The
class 4 rule grows but plateaus with the number of levels. We expect to see a
plateau once the size of the variables $X^\ell$ and $Y^\ell$ fully capture
the largest structures, and so the mutual information should become constant.
This shows that complexity does not grow via compositions of order from order.
Interposing disorder somewhere near the level of the largest structures would
result in further complexity growth.

The class 3 ECA is the only one which exhibits a qualitative difference between
the two figures. This is simply a finite sample statistical artifact which is
instructive to elucidate. In the $N\!=\!200$ case, we do not have enough states to
estimate the mutual information at the highest level ($\ell\!=\!7$) accurately.
The mutual information consists of two terms, the entropy of the single
variable $H(X^\ell)$ and the conditional entropy between the two variables
$H(X^\ell \mid Y^\ell)$. The entropy requires fewer samples to estimate
accurately than the conditional entropy. For example $X^7$ can take on $2^7$
possible values whereas the pair $(X^7, Y^7)$ admits $2^{14}$ distinct values.
With insufficient sample sizes, the conditional entropy shrinks prior to the
entropy and the mutual information is artificially inflated. This can be
remedied by increasing $N$. However, a parametric or otherwise more sample
efficient model for $P(X)$ and $P(X \mid Y)$ is an alternative that will likely
scale more robustly with the number of values of $X^\ell$.


\bibliographystyle{plainnat}
\bibliography{references}

\end{document}
