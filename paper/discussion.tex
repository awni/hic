\section{Discussion}

As simulations of digital life scale, robust and easy to implement measures
which can automatically detect emergent complexity will be critical.
Hierarchical information content is one such measure which can distinguish
systems with structural complexity from both purely random and highly ordered
systems. In other words, HIC exhibits the single hump in between order and
disorder required of any useful measure of complexity (see
fig.~\ref{fig:complexity_and_entropy}).

One of the benefits of HIC over some alternative measures of complexity is that
it requires very few assumptions to be used with a system. At most the levels
of the system and the sub-modules which make up a given level must be
determined. For systems which exhibit a natural hierarchy, HIC can be deployed
directly. We argued in section~\ref{sec:hierarchy_and_complexity} that
hierarchical structure drives complexity growth, and hence many naturally
occurring complex systems are hierarchical.

We've shown that HIC can distinguish complex elementary cellular automata from
random and highly ordered ones. Further studies should validate that HIC can
distinguish complexity in other artificial and natural systems. For example we
aim to apply HIC to two-dimensional cellular automata and to artificial
chemistries. We also intend to observe how well HIC captures the complexity of
natural systems, including for example protein networks or DNA sequences.

In many cases HIC on its own is sufficient, particularly when a system exhibits
a natural hierarchy and when measuring process complexity is not important.
However, we encourage researchers to consider HIC as one more tool in the
complexity measuring toolkit. For example, HIC and statistical complexity
together could capture both process and structural complexity which would be
difficult for either to capture when used in isolation.

Ultimately any purely quantitative measure of complexity should be used
carefully. Often objective functions that can be optimized numerically are
proxies for the intended goal. Prior to using a proxy one must validate that it
does indeed correlate with the intended goal. In the same way HIC, and any
other measure of complexity, should be validated prior to use with any new
system. Removing all subjectivity in assessing the complexity of a system is
difficult. A system which appears random to one observer can be meaningful
to another in the same way that the interpretation of a string of bits
depends on the recipient. In these cases, encoding prior information about the
structure of the system in the states used to compute the HIC will make it more
useful as a measure of complexity.

\citet[chap. 23]{hamming1997art} correctly predicted that ``we will...want
Mathematical models in which the whole is not the sum of the parts, but the
whole may be much more due to the `synergism' between the parts''. We are still
at the beginning of understanding the nature of complexity particularly as it
emerges from the interaction of simpler subsystems. Our understanding of
complex systems will grow through the use of tools like hierarchical
information content and other measures of complexity. We hope this will put
researchers in a better position in the future to formulate a more unified
theory of complexity.
