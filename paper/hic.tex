\section{Hierarchical Information Content}

We construct a measure for the structural complexity of a system. As motivated
in section~\ref{}, the complexity should grow through the compositions of
ordered and disordered subsystems in a hierarchical structure. At each such
composition (whether it by order to disorder or disorder to order), the
complexity grows. Hence we decompose the overall complexity as the sum of the
complexity at each such composition.

The complexity at each composition is given by the difference in mutual
information between the higher-level system and that of the subsystem. We
denote by $X^{\ell}$ the state variable at the $\ell$-th level of the
hierarchy. The neighborhood of $X^\ell$ is the set $\calN(X^\ell)$ which
contains the elements used to construct the system at the next level. Following
\citet{simon1991architecture}, we define the span of a level $d^\ell =
\left|\calN(X^\ell)\right|$ as the number of elements in the neighborhood. For
now we assume the span to be constant with a value of two $d^\ell = 2$ for
every level. We later generalize to larger spans.

The complexity at the composition of level $\ell$ from level $\ell -1$ is
computed from the squared difference of the mutual information of the
neighborhood at each level: 
\begin{equation}
C_\ell = \left( I(X^\ell, Y^\ell) - I(X^{\ell-1}, Y^{\ell-1}) \right)^2
\end{equation}
The mutual information can be computed from the Shannon entropy and conditional
entropy when $X^\ell$ is discrete or the differential entropy when $X^\ell$ is
continuous~\citep{cover1999elements}:
\begin{equation}
I(X^\ell, Y^\ell) = H(X^\ell) - H(X^\ell \mid Y^\ell) 
\end{equation}

We use the level complexities $C_\ell$ to construct the overall system complexity. 
\begin{definition}[Hierarchical Information Content]
\label{def:hic}
  The Hierarchical Information Content (HIC) of a hierarchical system $S$ with
  $L$ levels is given by:
  \begin{equation}
    \label{eq:hic}
    \textrm{HIC}(S) = \sum_{l=1}^L C_\ell = \left( I(X^\ell, Y^\ell) - I(X^{\ell-1}, Y^{\ell-1}) \right)^2.
  \end{equation}
\end{definition}
