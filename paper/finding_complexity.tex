\section{Finding Complexity}
\label{sec:finding_complexity}

We hypothesize that complexity grows from the compositions of order and
disorder. When a system composes an ordered structure with a disordered
collection of subsystems, opportunity for complexity growth emerges. Ordered
structures can also become disordered in subsequent higher levels. This
leads to a hierarchy of levels with varying degrees of order and disorder
creating a complex macro structure. We motivate the idea of inducing complexity
from composing order and disorder in the sequel followed by a motivation of the
importance of hierarchies.

\subsection{Order and Disorder}

\citet{adami2002complexity} defines two types of complexity: i) \emph{structural
complexity}, which can be thought of as complexity in space and ii)
\emph{process complexity}, which can be thought of as complexity in time.
Following \citet{schrodinger1944}, we consider two types of composition. The
first is order from order, and the second is order from disorder.  \citet[chap.
6]{schrodinger1944} argues that living beings exhibit order from order. In an
abstract sense, a living organism ``feeds on negative entropy'' (\emph{i.e.}
order) to produce the necessary building blocks needed to reverse the organisms
natural tendency towards entropy. More concretely, organisms create highly
structured subsystems at the molecular level (DNA, RNA, proteins, \emph{etc.})
by metabolising highly structured inputs. This process describes the
development of complexity through time. Hence complexity in time (process
complexity) involves the composition of order from order.

On the otherhand, complexity in space constructs higher-level structures from
lower level structures. If the structures are ordered throughout the levels of
the hierarchy, the global structure is unlikely to be complex. Some degree of
disorder should be present at various levels of the hierarchy. Thus complexity
in space (structural complexity) emerges in part from the composition of order
from disorder. Take as an example a binary tree and a tree found in nature.
Both have a fractal branching structure, but the latter is evidently much more
complex. We argue a feature of this complexity is the disorder found at the
varying levels of the hierarchy.

Given the distinction between how complexity manifests in time and space, two
separate measures may be appropriate. Here, we are solely concerned with the
latter -- measuring complexity in space. The nexus of complexity growth in space
is found in the composition of order from disorder. In order to continue to
grow in complexity, a structure must continue to increase the number of such
compositions. A hierarchy must develop with varying degrees of order and
disorder at varying levels. This is the primary motivation of our measure of
complexity which we define more rigorously in section~\ref{sec:hic}.

\subsection{Hierarchical Systems and Complexity}
\label{sec:hierarchy_and_complexity}

Many complex systems found in nature, both biotic and abiotic, have a
hierarchical structure. For example, the hierarchical structure of a cell
proceeds from atoms to molecules, then to marcomolecules, then
to organelles. Cells themselves aggregate into multicellular constructs which form
organs which result in human beings. Social systems almost always have
hierarchical structure. A complex business consists of individual workers,
teams, departments, business units, subsidiaries, and finally the company itself.
The galactic system is structured as a hierarchy consisting of
planets, solar systems, star clusters, galaxies, and even galactic clusters.

Following \citet{simon1991architecture}, we give an explanation for why complex
systems in nature are much more likely to be hierarchical than not.
\citet{simon1991architecture} gives as an example two watchmakers each using a
distinct assembly process. The former assembles the entire watch as a single
component. The latter makes a hierarchy of modules and assembles them into the
watch. If either watchmaker is interrupted they must begin the component they
were working on from scratch. \citet{simon1991architecture} shows, under
reasonable assumptions, that the watchmaker using sub-components has a
substantially greater probability of completing a watch than the watchmaker
using a single component.

More generally, emergent and stable structure must develop faster than the rate
of detioration driven by the second law of thermodynamics. A small number of
detiorations can cause a catastrophic failure which means the assembly process
must begin from scratch. Any attempt to assemble a complex structure \emph{de
novo} which is stable only upon completion is unlikely to succeed. Rather, a
hierachical assembly through intermediate stepping stones of stable subsystems
is more likely to result in the succesful assembly of the full system. If at
any stage a detioration occurs, the assembly need not start from scratch but
can draw on existing already assembled subsystems. The stable subsystems can be
assembled quickly from their own constituent susbsytems, so they too are less
likely to be thwarted by increasing entropy.

Similarly, \citet{cairns1995complexity} argues that since evolution
proceeds by small mutations, a scaffold of sorts is required to develop complex
structures. Once the complex structure is functional (\emph{i.e.} provides a
reproductive benefit), the scaffold can be lost without harm. With the scaffold
lost, evolution must proceed in a forward, complexity increasing direction to
further optimize an organism in a given niche. \citet{cairns1995complexity}
uses as an example an arch which needs a scaffold to build. However, once
built, the scaffold may be removed as the arch can stand on its own. This
so-called ``complexity ratchet'' gives both a mechanism to enable complexity
growth and an explanation for why the growth appears to be monotonic. In terms
of hierarchical structure, the scaffolding can be thought of as a structure
which holds together subsystems. When a reproductive advantage emerges from the
combination of the subsystems, the scaffolding can fall away resulting in a
more complex and hierarchical structure.

The process of using subsystems to create more complex larger systems is well
documented in evolution and the history of life. For example most of the
``major transitions''\footnote{These include the compartmentalization of
molecules, the hypothesized aggregation of RNA into Chromosomes, the
hypothesized symbiosis of two Prokaryotic cells into the Eukaryotic cell,
unicellular to multicellular organisms, individuals to colonies, and small
human groups to human societies.} in the evolution of life involve the
construction of a higher-level organism from lower level
components~\citep{smith1997major}. Using the fossil record as evidence,
\citet{mcshea2001hierarchical} demonstrates an increase in the maximum
hierarchical structure (the number of nested levels) found in organisms over
the course of evolution.\footnote{Curiously \citet{mcshea2001hierarchical} also
finds evidence for an increase in the rate of increase in the maximum
hierarchical structure. An optimistic implication is that an increase in
hierarchical complexity leads to an increase in the evolvability of an
organism.}

Complexity via composition and hierarchy has been demonstrated in simulations
of artificial chemistries~\citep{rasmussen2001ansatz, sayama2019cardinality}.
\citet{rasmussen2001ansatz} state as an ansatz (a trial solution which aids in
a more formal derivation) that a sufficiently complex set of subsystems is both
necessary and sufficient for hierarchical complexity to emerge.
\citet{sayama2019cardinality} suggests that the composition of lower-level
primitives into higher-level structure results in a ``cardinality leap''. For
example, arbitrary combinations of a finite set of building blocks yields a
countably-infinite set of higher-level structures. This provides an explanation
for why higher-level structure can be correlated with an increase in
complexity, particularly in the presence of natural selection.
