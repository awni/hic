\section{Finding Complexity}
\label{sec:finding_complexity}

We hypothesize that complexity grows from the compositions of order and
disorder.  When a system composes an ordered structure a disordered collection of 
subsystems, opportunity for complexity growth emerges. At a higher level many
ordered structures can become disordered. This leads to a hierarchy of levels
with varying degrees of order and disorder creating a complex macro structure.
We motivate the idea of inducing complexity from composing order and disorder
in the sequel, followed by a motivation of the importance of hierarchies. 

\subsection{Order and Disorder}

\citet{adami2002complexity} define two types of complexity: i) \emph{structural
complexity}, which can be thought of as complexity in space and ii)
\emph{process complexity}, which can be thought of as complexity in time.
Following \citet{schrodinger1944}, we consider two types of composition. The
first is order from order, and the second is order from disorder.  \citet[chap.
6]{schrodinger1944} argues that living beings exhibit order from order. In an
abstract sense, a living organism ``feeds on negative entropy'' (\emph{i.e.}
order) to produce the necessary building blocks needed to reduce the organisms
natural tendency towards entropy. More concretely, organisms create highly
structured subsystems at the molecular level (DNA, RNA, proteins, \emph{etc.})
by metabolising highly structured inputs. This process describes the
development of complexity through time. Hence complexity in time (process
complexity) involves the composition of order from order. 

On the otherhand, complexity in space constructs higher-level structures from
lower level structures. If the structures are ordered throughout the levels of
the hierarchy, the global structure is unlikely to be complex. Some degree of
disorder should be present at various levels of the hierarchy. Thus complexity
in space (structural complexity) emerges in part from the composition of order
from disorder.

Given the distinctions between how complexity manifests in time and space two
separate measures may be appropriate. Here, we are solely concerned with the
later -- measuring complexity in space. The nexus of complexity growth in space
is found in the composition of order from disorder. In order to continue to
grow complexity a structure must continue to increase the number of such
compositions. A hierarchy must develop where some levels are more ordered and
some are more disorderd. This is the primary motivation of our measure of
complexity which we define more rigorously in section~\ref{sec:hic}.

\subsection{Hierarchical Systems and Complexity}

Many complex systems found in nature, both biotic and abiotic, have a
hierarchical structure. For example, the hierarchical structure of a cell
begins follows in succession from atoms to molecules, then marcomolecules, then
organelles. Cells themselves aggregate into multicellular constructs which form
organs which result in human beings. Social systems almost always have
hierarchical structure. A complex business consists of individual workers,
teams, departments, business units, subsidiaries, and finally the full company.
The abiotic galactic system is structured as a hierarchy consisting of
planets, solar systems, star clusters, galaxies, and even galactic clusters.

Following \citet{simon1991architecture}, we give an explanation for why complex
systems in nature are much more likely to be hierarchical than not. Emergent
and stable structure must devlop faster than the entropy increasing tendency of
nature. An attempt to assemble a complex structure which begins from scratch
and is stable only upon completion is unlikely to succeed. More likely to work
is an assembly through the stepping stones of intermediate and stable
subsystems. The stable subsystems can be assembled quickly, so they are less
likely to be thwarted by increasing entropy. The subsystems can be used to
rapidly assemble a larger more complex stable system.

\citet{cairns1995complexity} argues that since evolution proceeds by small
mutations a scaffold of sorts is required to develop complex structures. Once
the complex structure is functional the scaffold can be lost without harm. With
the scaffold lost, evolution must proceed in a forward, complexity increasing
direction to further optimize an organism to a niche. This so-called
``complexity ratchet'' gives both a mechanism to enable complexity growth and
an explanation for why the growth appears to be monotonic. In terms of
hierarchical structure, the scaffolding can be thought of as a structure which
holds together subsystems.  When a reproductive advantage emerges from the
combination of the subsystems, the scaffolding can fall away resulint in a more
complex and hierarchical structure.

The process of using subsystems to create more complex larger systems is well
documented in evolution and the history of life. For example most of the
``major transitions''\footnote{These include the compartmentalization of
molecules, the hypothesized aggregation of RNA into Chromosomes, the
hypothesized symbiosis of two Prokaryotic cells into the Eukaryotic cell,
unicellular to multicellular organisms, individuals to colonies, and small
human groups to human societies.} in the evolution of life involve the
construction of a higher-level organism from lower level
components~\citep{smith1997major}. Using the fossil record as evidence,
\citet{mcshea2001hierarchical} demonstrates an increase in the maximum
hierarchical structure (the number of nested levels) found in organisms over
the course of evolution.\footnote{Curiously \citet{mcshea2001hierarchical} also
finds evidence for an increase in the rate of increase in the maximum
hierarchical structure. An optimistic implication is that an increase in
hierarchical complexity leads to an increase in the evolvability of an
organism.}  

Complexity via composition and hierarchy has been demonstrated in simulations
of artificial chemistries~\citep{rasmussen2001ansatz, sayama2019cardinality}.
\citet{rasmussen2001ansatz} state as an ansatz that a sufficiently
complex set of subsystems is both necessary and sufficient for hierarchical
complexity to emerge. \citet{sayama2019cardinality} suggests that the
composition of lower-level primitives into higher-level structure results in a
``cardinality leap''. A finite set of building blocks can yield a countably
infinite set of higher-level structures. This provides an explanation for why
higher-level structure can be correlated with an increase in complexity,
particularly in the presence of natural selection.
