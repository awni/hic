\section{Finding Complexity}

We hypothesize that complexity grows from the compositions of order and
disorder.  When a system composes an ordered structure from disordered
primitives, opportunity for complexity growth emerges. At a higher level many
ordered structures can become disordered. This leads to a hierarchy of levels
with varying degrees of order and disorder creating a complex macro structure.
We motivate the idea of inducing complexity from composing order and disorder
in the sequel, followed by a motivation of the importance of hierarchies. 

\subsection{Order from Disorder}

\citet{adami2002complexity} define two types of complexity: i) {\emph
structural complexity}, which can be thought of as complexity in space and ii)
{\emph process complexity}, which can be thought of as complexity in time.
Following \citet{schrodinger1944}, we consider two types of composition, the
first is order from order and the second is order from disorder.  \citet[chap.
6]{schrodinger1944} argues that living beings exhibit order from order. In an
abstract sense, an living organism ``feeds on negative entropy'' (\emph{i.e.}
order) to produce the necessary building blocks needed to reduce the organisms
natural tendency towards entropy. More concretely, organisms create highly
structured building blocks at the molecular level (DNA, RNA, proteins,
\emph{etc.}) by metabolising highly structured inputs. This process describes
the development of complexity through time. Hence complexity in time (process
complexity) is involves the composition of order from order. 

On the otherhand, complexity in space constructs higher-level structures from
lower level structures. If the structures are ordered throughout the levels of
the hierarchy, the global structure is unlikely to be complex. Some degree of
disorder should be present at various levels of the hierarchy. Thus complexity
in space (structural complexity) emerges in part from the composition of order
from disorder.

Given the distinctions between how complexity manifests in time and space two
separate measures may be appropriate. Here, we are concerned with the later --
measuring complexity in space. The nexus of complexity growth is found in the
composition of order from disorder. In order to continue to grow complexity a
structure must continue to increase the number of such compositions. A
hierarchy must develop where some levels are more ordered and some are more
disorderd. This it the primary motivation of our measure of complexity which we
define more rigorously in section~\ref{sec:hic}.

\subsection{Complexity in Hierarchies}

We now consider the importance of hierarchy in complexity. inducing composition
of order from disorder
